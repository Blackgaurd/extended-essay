% latex template
\documentclass[12pt]{article}
\usepackage[utf8]{inputenc}
\usepackage[english]{babel}
\usepackage[margin=1in]{geometry}
\usepackage{caption} % figure captioning
\usepackage{amsmath} % equation tools
\usepackage{amssymb} % math symbols
\usepackage{mathtools} % math rendering improvements
%\usepackage{siunitx} % standard units
\usepackage{graphicx} % add images
%\usepackage{wrapfig} % position images
\usepackage{float} % position images pt.2
%\usepackage[makeroom]{cancel} % cross out text
%\usepackage[version=4]{mhchem} % chemical equations
%\usepackage{multicol} % multiple columns
%\usepackage{pgfplots} % built in plotter
%\pgfplotsset{width=10cm,compat=1.9} % plotter settings
\usepackage{hyperref} % table of contents links
\usepackage{indentfirst} % indent first paragraph after section

\hypersetup{
    colorlinks,
    citecolor=black,
    filecolor=black,
    linkcolor=black,
    urlcolor=black
}

\renewcommand{\baselinestretch}{1.5} % line spacing
\newcommand{\fline}{\par\noindent\rule{\textwidth}{0.1pt}} % horizontal line (wide)

\title{EE TITLE}
\author{Bryan Deng}

\begin{document}

\maketitle
\newpage
\tableofcontents
\newpage

\section{Background Information}

\subsection{Machine Learning and its Applications}

Machine learning is a branch of artificial intelligence that uses large datasets and algorithms to mimic the way humans learn and improve accuracy over time \cite{whatisml}. Since its debut in 1952, it has been steadily gaining popularity and recognition for its abilities in recognizing patterns and continuous learning.

\section{Experimental Methodology}

\section{Data Analysis}

\section{Error Analysis}

\section{Conclusion}

\section{Appendix}

\bibliographystyle{plain}
\bibliography{refs}

\end{document}

